%%%%%%%%%%%%%%%%%%%%%%%%%%%%%%%%%%%%%%%%%
% Medium Length Professional CV
% LaTeX Template
%
% This template has been downloaded from:
% http://www.LaTeXTemplates.com
%
% Original author:
% Trey Hunner (http://www.treyhunner.com/)
%
% Important note:
% This template requires the resume.cls file to be in the same directory as the
% .tex file. The resume.cls file provides the resume style used for structuring the
% document.
%
%%%%%%%%%%%%%%%%%%%%%%%%%%%%%%%%%%%%%%%%%

%----------------------------------------------------------------------------------------
%	PACKAGES AND OTHER DOCUMENT CONFIGURATIONS
%----------------------------------------------------------------------------------------

\documentclass{resume} % Use the custom resume.cls style

\usepackage[left=0.75in,top=0.7in,right=0.75in,bottom=0.6in]{geometry} % Document margins
\usepackage{gensymb}

\name{Michael Bella} % Your name
\address{88 E. San Fernando St Unit 711, San Jose, CA 95113} % Your address
\address{408~-~717~-~0367 \\ michael.j.bella@gmail.com} % Your phone number and email

\begin{document}

%----------------------------------------------------------------------------------------
%	WORK EXPERIENCE SECTION
%----------------------------------------------------------------------------------------

\begin{rSection}{Work Experience}

\begin{rSubsection}{KLA-Tencor}{December 2011 - Present}{Electrical Engineer}{Milpitas, CA}

\begin{rWorkProject}{350\celsius \space Calibration System}
\item Rewired existing high temperature oven to be controlled by a National Instruments Compact FieldPoint.
\item Developed state chart for control of both oven systems
\item Tuned cascaded control loops to bring the calibration system up to each temperature set point with minimal overshoot.
\end{rWorkProject}

\begin{rWorkProject}{High Temperature Wireless Wafer}
\item Wrote LabView which automatically tests all possible failure modes of a wafer substrate with thin-film aluminum traces.
\item Wrote LabView application to use an Agilent 3490a as a curve tracer to manually diagnose faults in substrates
\item Designed a lump element model of the inductive wafer charging system in order to implement a simulated wafer communication system.
\item Hand wired prototype of a nanoamp current measurement fixture.
\item Hand wired interface board prototypes to enable the test and calibration of these wafer prototypes
\end{rWorkProject}

\begin{rWorkProject}{New Sensorised Wafer Project}
\item Modified existing embedded C wafer code base to work with new types of sensors.
\begin{itemize}
\itemsep -0.5em \vspace{-0.5em}
\renewcommand{\labelitemi}{-}
\item Rewrote measurement subsystem to interface with the new sensor type
\item Redesigned data-store format to support the new sensor
\item Fully tested all code changes against existing low power specifications for the product family.
\item Rewrote portions of existing Mfg software to support calibration
\end{itemize}

\item Wrote PC software in LabView to launch wafer missions, and to retrieve data from these new sensors.
\item Designed custom data processing software in Python to support data driven development of new sensor platforms.
\end{rWorkProject}

\begin{rWorkProject}{FOUP Improvments}
%\item Designed an improved detector circuit to recover the sensor wafer's data stream
\item Designed a circuit to recover the wafer communication signal from the envelope detector with improved sensitivity over the existing design.
\item Characterized the improved wafer communication system performance.
\end{rWorkProject}

\smallskip
\item Designed many different automated test and measurement applications in LabView.
\item Wrote LabView software to acquire and process data from two spectrometers.
\item Designed triggering system to enable the simultaneous capture data capture with two spectrometers.
\end{rSubsection}

%------------------------------------------------

\begin{rSubsection}{KLA-Tencor Internship}{June 2005 - December 2011}{Electrical Engineer}{Milpitas, CA}
%\item Wrote automated test and measurement applications in LabView for a wide range of projects.
\item Developed LabView code for a wide range different projects
\begin{itemize}
\itemsep -0.5em \vspace{-0.5em}
\renewcommand{\labelitemi}{-}
\item Automated capacitor tester
\item Wireless communication system tester
\item Synchronous serial link to a custom embedded sensor system
\end{itemize}
\item Characterized the magnetically coupled wafer communication system
\item Performed PCB/PCA diagnostic work, failure analysis, rework of SMD and through hole components.
\end{rSubsection}

\end{rSection}

%----------------------------------------------------------------------------------------
%	TECHNICAL STRENGTHS SECTION
%----------------------------------------------------------------------------------------
\pagebreak
\begin{rSection}{Technical Strengths}

\begin{tabular}{ @{} >{\bfseries}l @{\hspace{6ex}} l }
Programming Languages & Embedded C, LabView, Python, C/C++  \\ \\
Tools & Eclipse, git, SVN, Code Composer Studio, IAR, Spice, AWR Microwave Office, \\
 & CADSoft Eagle\\ \\
Design Experience & Low Power Embedded Systems, RF Matching Networks \& Amplifiers\\
 & Analog Signal Processing, High Precision Analog Measurement, SMPS Design\\ \\
 
Lab Skills & SMD Soldering, Wiring harness construction, PCA Bringup and Debug\\
 & Prototyping, Build designs from print \\ \\
 
Other & Proficient with Linux, Texas Instruments MSP430 \\
\end{tabular}

\end{rSection}

%----------------------------------------------------------------------------------------
%	EDUCATION SECTION
%----------------------------------------------------------------------------------------

\begin{rSection}{Education}

{\bf San Jose State University} \hfill {\em December 2011} \\ 
B.S. in Electrical Engineering \\

\end{rSection}

%----------------------------------------------------------------------------------------
%	EXAMPLE SECTION
%----------------------------------------------------------------------------------------

%\begin{rSection}{Section Name}

%Section content\ldots

%\end{rSection}

%----------------------------------------------------------------------------------------

\end{document}
