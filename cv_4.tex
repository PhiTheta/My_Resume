%%%%%%%%%%%%%%%%%%%%%%%%%%%%%%%%%%%%%%%%%
% Medium Length Professional CV
% LaTeX Template
%
% This template has been downloaded from:
% http://www.LaTeXTemplates.com
%
% Original author:
% Trey Hunner (http://www.treyhunner.com/)
%
% Important note:
% This template requires the resume.cls file to be in the same directory as the
% .tex file. The resume.cls file provides the resume style used for structuring the
% document.
%
%%%%%%%%%%%%%%%%%%%%%%%%%%%%%%%%%%%%%%%%%

%----------------------------------------------------------------------------------------
%	PACKAGES AND OTHER DOCUMENT CONFIGURATIONS
%----------------------------------------------------------------------------------------

\documentclass{resume} % Use the custom resume.cls style

\usepackage[left=0.7in,top=0.4in,right=0.7in,bottom=0.5in]{geometry} % Document margins
\usepackage{gensymb}
\usepackage{needspace}

\name{Michael Bella} % Your name
\address{7375 Rollingdell Dr. Cupertino, California 95014} % Your address
\address{408~-~717~-~0367 \\ michael.j.bella@gmail.com} % Your phone number and email

\begin{document}

%----------------------------------------------------------------------------------------
%	TECHNICAL STRENGTHS SECTION
%----------------------------------------------------------------------------------------
%\pagebreak
\begin{rSection}{Technical Strengths}

\begin{tabular}{ @{} >{\bfseries}l @{\hspace{6ex}} l }
Programming Languages & Python, C, LabView  \smallskip \\

Software Tools & Eclipse, Git, SVN, Code Composer Studio, IAR, Spice\\ 
 & AWR Microwave Office, CADSoft Eagle, Matlab \smallskip \\

Design Experience & Low Power Embedded Systems, RF Matching Networks \& Amplifiers\\
 & Analog Signal Processing, High Precision Analog Measurement, SMPS Design \smallskip \\
 
Lab Skills & Root Cause Analysis, SMD Soldering, Wiring harness construction,\\
& PCA Bringup and Debugging, Prototyping, Build designs from print\smallskip \\
 
Other Technical Experience &  Low Power System Design, I2C, SPI, JTAG, Boundary Scan \\
\end{tabular}

\end{rSection}
\medskip

%----------------------------------------------------------------------------------------
%	WORK EXPERIENCE SECTION
%----------------------------------------------------------------------------------------

\begin{rSection}{Work Experience}
\vspace{-0.5em}
\begin{rSubsection}{Apple Inc. -- Hardware Test Engineering}{October 2013 - Present}{Electrical Engineer}{Cupertino, CA}

\item Design and impliment test plans for component level verification on NPI projects
\begin{itemize}
\itemsep -0.5em \vspace{-0.5em}
\renewcommand{\labelitemi}{-}
\item Manage test vendors to ensure that all tests are properly implimented.
\item Develop python scripts for test automation and data processing.
\item Identify test line issues and quickly drive them to root cause.
\end{itemize}
%\medskip
\end{rSubsection}

%----------------------------------------------------------------------------------------

\begin{rSubsection}{KLA-Tencor -- SensArray Group}{December 2011 - October 2013}{Electrical Engineer}{Milpitas, CA}

\item Developed production code for ultra low power MSP430 based systems.
\begin{itemize}
\itemsep -0.5em \vspace{-0.5em}
\renewcommand{\labelitemi}{-}
\item Designed embedded systems to serve as platforms for new sensor technologies.
\item Adapt existing measurement system architectures for use with new sensors.
\item Modify existing embedded system code bases to work with new types of sensors.
\item Characterize and test RFID systems for use in ultra low power embedded applications.
\end{itemize}
%\medskip

\item Designed test fixtures for both production and R\&D use.
\begin{itemize}
\itemsep -0.5em \vspace{-0.5em}
\renewcommand{\labelitemi}{-}
\item Developed firmware and software to communicate with and process data from several types of sensors.
\item Characterized sensors, ASICs, and passive components for use in new product designs.
\item Tested the functionality of sensor ICs at different steps in their processing \& assembly.
\item Desiged and tuned RF matching networks for use in high power and plasma systems.
\end{itemize}
%\medskip

\end{rSubsection}

%----------------------------------------------------------------------------------------
\pagebreak[2]
\begin{rSubsection}{KLA-Tencor/SensArray Internship}{June 2005 - December 2011}{Electrical Engineering Intern}{Milpitas, CA}
%\item Wrote automated test and measurement applications in LabView for a wide range of projects.
\item Debugged and performed FA on low power embedded systems under the guidance of Senior System Engineers. 
\item Performed PCB/PCA diagnostic work and repair, failure analysis, SMD rework.
\item Developed LabView code to interface with test equipment and embedded systems for automated testing and R\&D.
\end{rSubsection}

\end{rSection}

%----------------------------------------------------------------------------------------
%	Personal & Student Projects SECTION
%----------------------------------------------------------------------------------------
\pagebreak[3]
\begin{rSection}{Personal \& Student Projects}
\vspace{-0.5em}
\begin{rProject}{Bike Light}{}
\item Designed a bikelight controller to perform battery monitoring and control RGB LED arrays.
\item Calculated power budget based on available power sources.
\item Selected appropriate LED drivers for the application.
\end{rProject}
\vspace{-0.5em}
\begin{rProject}{Kite Control System for Wind Power Generation}{2013 - Present}
\item Started this project as part of at team at the first Makathon competition (www.makathon.org)
\item Designed rigging to control power kite lines 
\item Developing python code to detect the kite with a webcam using openCV, and control 
\item Developing python code to interface with Logosol brushless motor controller
\end{rProject}
\vspace{-0.5em}
\begin{rProject}{SJSU Formula Hybrid Vehicle Team}{2010 - 2011}
\item Developed firmware for a PIC based battery management system.
\item Helped teammates debug issues with their switching power coverter.
\end{rProject}
\end{rSection}

%----------------------------------------------------------------------------------------
%	EDUCATION SECTION
%----------------------------------------------------------------------------------------
\pagebreak[3]
\begin{rSection}{Education}

{\bf San Jose State University} \hfill {\em December 2011} \\ 
B.S. in Electrical Engineering \\

\end{rSection}

%----------------------------------------------------------------------------------------
%	EXAMPLE SECTION
%----------------------------------------------------------------------------------------

%\begin{rSection}{Section Name}

%Section content\ldots

%\end{rSection}

%----------------------------------------------------------------------------------------

\end{document}
