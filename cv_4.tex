%%%%%%%%%%%%%%%%%%%%%%%%%%%%%%%%%%%%%%%%%
% Medium Length Professional CV
% LaTeX Template
%
% This template has been downloaded from:
% http://www.LaTeXTemplates.com
%
% Original author:
% Trey Hunner (http://www.treyhunner.com/)
%
% Important note:
% This template requires the resume.cls file to be in the same directory as the
% .tex file. The resume.cls file provides the resume style used for structuring the
% document.
%
%%%%%%%%%%%%%%%%%%%%%%%%%%%%%%%%%%%%%%%%%

%----------------------------------------------------------------------------------------
%	PACKAGES AND OTHER DOCUMENT CONFIGURATIONS
%----------------------------------------------------------------------------------------

\documentclass{resume} % Use the custom resume.cls style

\usepackage[left=0.75in,top=0.6in,right=0.75in,bottom=0.6in]{geometry} % Document margins
\usepackage{gensymb}

\name{Michael Bella} % Your name
\address{88 E. San Fernando St Unit 711, San Jose, CA 95120} % Your address
\address{(408)~$\cdot$~717~$\cdot$~0367 \\ michael.j.bella@gmail.com} % Your phone number and email

\begin{document}

%----------------------------------------------------------------------------------------
%	WORK EXPERIENCE SECTION
%----------------------------------------------------------------------------------------

\begin{rSection}{Work Experience}

\begin{rSubsection}{KLA-Tencor}{December 2011 - Present}{Electrical Engineer}{Santa Clara, CA}

\begin{rWorkProject}{350\celsius \space Calibration System}
\item Rewired an existing high temperature oven to be controlled by a National Instruments CompactFieldPoint.
\item Tuned cascaded control loops to bring the isothermal chamber up to each temperature set point with minimal overshoot.
\item Designed a lump element model of the inductive wafer charging system in order to implement a simulated wafer communication system.
%\item Designed an interface board for use in the calibration process of these High Temperature Wafers
\end{rWorkProject}

\begin{rWorkProject}{New Sensor Project}
\item Modified existing wafer code base to work with new types of sensors.

\item Tested modified code to ensure that all low power requirements are met.
\item Wrote PC software to launch wafer missions, and to retrieve data from these new sensors.
\item Designed custom data processing software in Python to support data driven development of new sensor platforms.
\end{rWorkProject}

\begin{rWorkProject}{FOUP Improvments}
\item Designed an improved detector circuit to recover the sensor wafer's data stream
\item Charactized 
\end{rWorkProject}

\item Designed many different automated test and measurement applications in LabView.
\item Wrote LabView software and designed triggering system to capture simultaneous data with two spectrometers.
\end{rSubsection}

%------------------------------------------------

\begin{rSubsection}{KLA-Tencor Internship}{June 2005 - December 2011}{Electrical Engineer}{San Jose, CA}
\item Wrote automated test and measurement applications in LabView for a wide range of projects.
\item Characterized the magnetically coupled wafer communication system 
\end{rSubsection}

\end{rSection}

%----------------------------------------------------------------------------------------
%	TECHNICAL STRENGTHS SECTION
%----------------------------------------------------------------------------------------

\begin{rSection}{Technical Strengths}

\begin{tabular}{ @{} >{\bfseries}l @{\hspace{6ex}} l }
Programming Languages & Embedded C, LabView, Python, C/C++  \\
Tools & Eclipse, git, SVN, Code Composer Studio, IAR, Spice, AWR Microwave Office \\
Design Experience & Low Power Embedded Systems, RF Matching Networks \& Amplifiers\\
 & Analog Signal Processing, High Precision Analog Measurement, SMPS Design\\

\end{tabular}

\end{rSection}

%----------------------------------------------------------------------------------------
%	EDUCATION SECTION
%----------------------------------------------------------------------------------------

\begin{rSection}{Education}

{\bf San Jose State University} \hfill {\em December 2011} \\ 
B.S. in Electrical Engineering \\

\end{rSection}

%----------------------------------------------------------------------------------------
%	EXAMPLE SECTION
%----------------------------------------------------------------------------------------

%\begin{rSection}{Section Name}

%Section content\ldots

%\end{rSection}

%----------------------------------------------------------------------------------------

\end{document}
