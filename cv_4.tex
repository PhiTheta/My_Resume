%%%%%%%%%%%%%%%%%%%%%%%%%%%%%%%%%%%%%%%%%
% Medium Length Professional CV
% LaTeX Template
%
% This template has been downloaded from:
% http://www.LaTeXTemplates.com
%
% Original author:
% Trey Hunner (http://www.treyhunner.com/)
%
% Important note:
% This template requires the resume.cls file to be in the same directory as the
% .tex file. The resume.cls file provides the resume style used for structuring the
% document.
%
%%%%%%%%%%%%%%%%%%%%%%%%%%%%%%%%%%%%%%%%%

%----------------------------------------------------------------------------------------
%	PACKAGES AND OTHER DOCUMENT CONFIGURATIONS
%----------------------------------------------------------------------------------------

\documentclass{resume} % Use the custom resume.cls style

\usepackage[left=0.65in,top=0.4in,right=0.65in,bottom=0.4in]{geometry} % Document margins
\usepackage{gensymb}
\usepackage{needspace}

\name{Michael Bella} % Your name
\address{Cupertino, California 95014} % Your address
\address{408~-~717~-~0367 \\ michael.j.bella@gmail.com} % Your phone number and email

\begin{document}

% Do a thing!
\hyphenpenalty=10000

%----------------------------------------------------------------------------------------
%	TECHNICAL STRENGTHS SECTION
%----------------------------------------------------------------------------------------
%\pagebreak
\begin{rSection}{Technical Strengths}

\begin{tabular}{ @{} >{\bfseries}l @{\hspace{6ex}} l }
Programming Languages & Python, C, LabView, Verilog \smallskip \\

Software Tools & Eclipse, Git, SVN, Code Composer Studio, IAR, Spice\\ 
 & AWR Microwave Office, CADSoft Eagle, Matlab, JMP \smallskip \\

Design Experience & Low Power Embedded Systems, High Precision Analog Measurement \\ 
 & Analog Design, PCB Layout, SMPS Design, RF Matching \& Amplifiers \smallskip \\
 
Lab Skills & Root Cause Analysis, SMD Soldering, Wiring Harness Construction,\\
 & PCA Bringup and Debugging, Prototyping, Build Designs from Print\smallskip \\
 
Other Technical Experience & I2C, SPI, JTAG, Boundary Scan \\
\end{tabular}

\end{rSection}

%----------------------------------------------------------------------------------------
%	WORK EXPERIENCE SECTION
%----------------------------------------------------------------------------------------
\vspace{0.5em}
\begin{rSection}{Work Experience}
\vspace{-0.5em}
\begin{rSubsection}{Apple Inc. -- Hardware Test Engineering}{October 2013 - Present}{Electrical Engineer}{Cupertino, CA}
\item Design and implement test plans for component and system level testing on iOS and accessory projects.
\item Manage test vendors and contract manufacturers to ensure efficient implementation of all required tests.
\item Work with cross functional engineering teams to identify test line issues and drive them to root cause.
\item Automate functional testing and data processing tasks using Python.
\end{rSubsection}

%----------------------------------------------------------------------------------------

\ssquish
\begin{rSubsection}{KLA-Tencor -- SensArray Group}{December 2011 - October 2013}{Electrical Engineer}{Milpitas, CA}
\item Rewrote scheduling and flash data storage code in order to use an existing code base with a new types of sensors.
\item Developed firmware for MSP430 family microcontrollers to evaluate new sensor types for customer application investigations.
\item Worked as part of a team to design a new embedded system architecture to lower power consumption, improve measurement accuracy, increase system flexibility, and increase product reliability.
\item Developed python code to extract step heights from data generated by research prototypes.
\item Wrote LabView applications to use test equipment such as Agilent 34410A, 34970A, oscilloscopes, and LCR meters.
\end{rSubsection}

%----------------------------------------------------------------------------------------
\pagebreak[2]
\ssquish
\begin{rSubsection}{KLA-Tencor/SensArray Internship}{June 2005 - December 2011}{Electrical Engineering Intern}{Milpitas, CA}
\item Debugged and performed failure analysis on low power embedded systems. 
\item Developed LabView applications to interface with test equipment and embedded systems for automated testing.
\end{rSubsection}
\end{rSection}

%----------------------------------------------------------------------------------------
%	Personal & Student Projects SECTION
%----------------------------------------------------------------------------------------
\pagebreak[3]
\begin{rSection}{Personal \& Student Projects}

\ssquish
\begin{rProject}{Kite Control System for Wind Power Generation}{2013 - Present}
\item Developing Python code to detect a kite using openCV and send commands to a Logosol motor controller.
\item Designing rigging to control a power kite using a servo or stepper motor.
\item Started this project as part of a team at the first Makathon competition (www.makathon.org).
\end{rProject}

\ssquish
\begin{rProject}{Bike Light - 1000 lm Headlamp and RGB Taillamp}{2012}
\item Designed and programmed a bikelight controller to perform battery monitoring and control RGB LED arrays.
\item Calculated power budget and selected appropriate LED drivers for my application.
\end{rProject}

\ssquish
\begin{rProject}{Formula Hybrid Vehicle Team - SJSU}{2010 - 2011}
\item Developed firmware for a PIC based battery management system.
\item Worked with teammates to debug high power switching converter issues.
\end{rProject}
\end{rSection}


%----------------------------------------------------------------------------------------
%	EDUCATION SECTION
%----------------------------------------------------------------------------------------
\pagebreak[3]
\begin{rSection}{Education}

{\bf San Jose State University} \hfill {\em December 2011} \\ 
B.S. in Electrical Engineering \\

\end{rSection}

%----------------------------------------------------------------------------------------
%	EXAMPLE SECTION
%----------------------------------------------------------------------------------------

%\begin{rSection}{Section Name}

%Section content\ldots

%\end{rSection}

%----------------------------------------------------------------------------------------

\end{document}
